\documentclass[a4paper, 11pt]{article}

\usepackage{kotex} % Comment this out if you are not using Hangul
\usepackage{fullpage}
\usepackage{hyperref}
\usepackage{amsthm}
\usepackage[numbers,sort&compress]{natbib}

\theoremstyle{definition}
\newtheorem{exercise}{Exercise}

\begin{document}
%%% Header starts
\noindent{\large\textbf{IS-521 Activity Proposal}\hfill
                \textbf{Yunjong Jeong}} \\
         {\phantom{} \hfill \textbf{blukat29}} \\
         {\phantom{} \hfill Due Date: April 15, 2017} \\
%%% Header ends

\section{Activity Overview}

장고 걸스 튜토리얼 (Django Girls Tutorial) \cite{djangogirls} 을 따라하면서 웹사이트를 스스로 제작해본다.
웹사이트 제작을 제안하는 이유는, 강의 계획에 나와있듯이 수업에서 웹 보안에 대해 다룰 에정인데,
활동을 통해 웹 어플리케이션과 HTTP에 대한 이해를 얻으면 수업 내용을 더욱 수월하게 이해할 수 있을 것으로 기대하기 때문이다.
장고 걸스 튜토리얼을 선택한 이유는, 설명이 굉장히 친절하고 내용에 군더더기가 없이 핵심만 빠르게 다루기 때문이다.

\section{Exercises}

\begin{exercise}

  파이썬과 장고 개발환경을 준비한다.

\end{exercise}

\begin{exercise}

  DB 모델과 템플릿을 제작한다.

\end{exercise}

\begin{exercise}

  로그인 기능을 구현한다.

\end{exercise}

\section{Expected Solutions}

잘 작동하는 웹페이지가 만들어진다.

\bibliography{references}
\bibliographystyle{plainnat}

\end{document}
